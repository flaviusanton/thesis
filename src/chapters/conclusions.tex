\chapter{Conclusions}

In this thesis we have presented a pipelined, distributed, social media image
processing infrastructure with exemplification on logo recognition. The
biggest objective we had and fulfilled was to build the system in an
algorithm agnostic manner, such that it supports any type of image processing
providing it has the implementation of an algorithm.

In order to have a fully functional, complete, system we have also implemented
a logo recognition algorithm that is based on pattern matching and better
described in \labelindexref{Section}{sec:logo-alg}. We chose this
approach for two main reasons. First, because it provides good results without
using too much computing power and second, because the amount of time
allocated for developing the algorithm was only a fraction of the total
time needed to build the project and we needed an algorithm that can be well
understood and implemented relatively fast.

Even with these numerous constraints, mainly presented in
\labelindexref{Section}{sec:reqs-tech}, the results we have obtained (see \labelindexref{Section}{sec:cost-and-resource} and
\labelindexref{Section}{sec:prec-recall}) are more than satisfactory and
we have a strong belief that the system can be well integrated in a real social media
analytics platform, like the one that Hootsuite provides.
