\chapter{Existing Work}

Like we saw in the previous chapter, social media analytics has become more
and more important, therefore the businesses centred around it have gained a
lot more attention because they produce a lot more impact. Over the past few
years, whole companies like Hootsuite or Sprout
Social\footnote{\url{http://sproutsocial.com}} emerged around the social
media ecosystem offering a large variety of tools not only for analytics, but
for optimizing social media experience in general. Not unexpectedly, gigants
like Google or Facebook have built tools that help their users to better
understand how they perform on their very own social network.

However, all these above solutions can be called rudimentary, at most, when it
comes to logos. Most of them do not even take logos into account, while others
have recently started to do some work in the domain. Of course, there is
already some progress done in the industry and it seems quite professional,
but its main disadvantage is that it is proprietary technology. The solution
that Ditto Labs, Inc\footnote{\url{http://ditto.us.com}} is promoting is one
of the very few in the industry and seems promising offering visual search in
social media streams, but it is not open source so we cannot say much about
their approach.

In the next section we will explore one of the most innovative applications of
image recognition in social media, while for the rest of this chapter we will
concentrate on two techniques of logo recognitions, one of them featuring many
written papers in the Computer Vision domain.

\section{Image Based Analytics in Social Media}

One of the new startups in the social media analytics industry is
Curalate\footnote{\url{http://www.curalate.com}}, which, according to their
claims, is the world's leading marketing and analytics suite for the visual
web. "Their business started around Pinterest and tried to analyze Pins for
their potential to expose campaigns and products. As they began to build the
business, they realized many people do not use text on Pinterest as they do on
other social media sites and therefore needed to figure out a way to
specifically analyze images on their
own"\footnote{\url{http://www.psfk.com/2014/06/curalate-social-image-analysis.html}}.

Curalate's main idea is that people started to prefer communicating using
images, especially on social networks dedicated for this, like Pinterest or
Tumblr, and so they try to use object recognition techniques to offer better
insight to their clients. Curalate CEO, Apu Gupta, said in an interview:
"Before people buy, they save an image of a product and pin it on Pinterest
or post it on Tumblr. After they buy, they might take a selfie with their new
sweater and post it on Instagram."

One true thing that stands for all the work done in the domain is that
none\footnote{To the best of our knowledge} of the companies involved released
an open source platform that allows multiple types of social media image
processing, not only logos or objects.

\section{Object and Logo Recognition}

Let us dive now in the realm of object recognition as a stand alone activity,
not paired up with social media or marketing purposes. Fortunately, the
Computer Vision is a well studied and researched field and there is constant
progress in the area. Before proceeding, let's be clear with the term
\textit{object recognition} which can be defined as the "process for
identifying a specific object in a digital image or video. Object recognition
algorithms rely on matching, learning, or pattern recognition algorithms using
appearance-based or feature-based techniques. Common techniques include edges,
gradients, Histogram of Oriented Gradients (HOG), Haar wavelets, and linear
binary patterns."\footnote{\url{http://www.mathworks.com/discovery/object-recognition.html}}

In the next sections, we will see some of the techniques used for object or
logo recognition together with advantages and disadvantages. Before that,
let's see what is different in recognizing a logo compared to recognizing an
object, because if there were no differences we could simply use any type of
object recognition approach and apply it in our case. 

The first one that comes to mid, if we look at
\labelindexref{Figure}{img:8-logos}, may probably be the \textit{color}. One of the most
definitive characteristics of a logo is that its color changes very rarely,
or, if it actually changes, we can almost consider it to be another logo. There are
brands that change their logo color drastically over time (i.e. Apple), but
that is not the common case. Another important fact about the color is that
not only there are few or inexistent changes, but there are few colors that
appear in the logo itself, to begin with. If we take the Twitter, Dropbox or
Honda logos, we can clearly see that they are made of one single color, maybe
with slight gradients.

\fig[scale=0.4]{src/img/8-logos.pdf}{img:8-logos}{Random Logos}

Even the Metro Goldwyn Mayer logo, which seems very different compared to the
other 7 logos, has one dominant color. Exceptions from this rule make Google
and the old Apple logo, but even though they comprise of several colors, these
are mostly primary and they do not overlap, so we can see a very abrupt
transition from one color to another, one that do not usually occurs in a
natural environment.

This observation brings us to the next major difference between a logo and a
random object, the \textit{shape}. While an object, despite being the same
type of object, can have different shapes and colors (imagine that we are
trying to do facial recognition), a brand's logo is always the same. It is one
of the key ideas in using a logo that it never has to change in order for
people to remember and recognize it easily, so we can also use it to our
advantage. Therefore, we can assume that a specific logo, like the Google one,
for example, will appear with the same font, the same aspect ratio, the
same amount of space between its letters and so on and so forth. In some
cases, it is even illegal to change the aspect of a logo, so we can rely on
this and use it to our advantage.

That being said, we can also do another important observation regarding logos:
they are very rarely, if at all, rotated. There are cases when we have to
analyze an image with a Coca Cola can that is flipped on one side so the logo
is also rotated 90 degrees, but these are situations that happen in less than
10\% of the images used to build this project. With this in mind, we can
safely state that considering logo rotation is not the biggest concern
regarding the scope of this thesis.

To sum up the above observations, we have seen that three features of a logo
usually stay the same across multiple occurrences: \textit{color},
\textit{shape} and \textit{orientation}. Taking this into consideration, when we
speak about logo recognition a natural idea should come to mind:
\textit{Template Matching}.

\subsection{Template Matching Approach}
\label{sub-sec:template-matching}

This technique is one of the simplest, yet effective ones used in this area of
study. In template matching, the query image is matched against an existing
template according to a similarity measure, for example the Euclidean
distance. "The templates can be represented as intensity or color images, when the
appearance of the object has to be considered. Appearance templates are often
specific and lack of generalization because the appearance of an object is
usually subject to the lighting condition and surface property of the object.
Therefore, binary templates representing the contours of the object are often
used in object detection since the shape information can be well captured by
the templates. Given a set of binary templates representing the object, the
task of detecting whether an image contains the object eventually becomes the
calculation of the matching score between each template and the
image."\cite{improved-template}

One place where we can find a good implementation of the template matching
algorithm is the open source computer vision library, OpenCV\cite{open-cv}.
We will see later, in \labelindexref{Chapter}{chapter:proposed-sol}, that we
are actually using the OpenCV C++ library within the logo detection component. As
\labelindexref{Figure}{img:sliding-template} shows, the main idea here is to
have a template smaller than the actual query image (in case we find ourselves
in the trivial case where the template is bigger than the query image it means
there is no possible match) and try and fit the template in every possible
position on the original image. If the query image is big enough and the
sliding box small compared to the former, it is not that hard to notice an
obvious flaw of this approach: it can be very slow in some cases. Fortunately,
in real life situations, there is usually no need for matching at high
resolutions, so both images can be safely resized to much smaller sizes as
long as we do not lose too many details, especially in the template image.

\fig[scale=1]{src/img/template-matching.jpg}{img:sliding-template}{Sliding
Template}

In order for the template matching approach to work, there is a need for a
good matching function. In a real implementation, like the one in the OpenCV
library, a third image, sometimes called a \textit{result image}, is obtained
after the sliding step and on that image we look for the \textit{best} match.
When we slide the template on the image, and by sliding we understand moving
the box one pixel at a time from left to right and up to down, we compute a
value for the same pixel in the result image based on a metric. So, for each
pair of pixels \((x, y)\) where we we overlapped the template image \(T\) over
the query image \(I\) we set the corresponding \((x, y)\) pixels in the result
image \(R\) according to the match metric.

The OpenCV library exposes several metrics to be used together with the
\texttt{matchTemplate} function of which four of the most interesting ones are:
\begin{itemize}
  \item \texttt{CV_TM_SQDIFF}
    \[R(x, y) = \sum_{(x', y') \in T} (T(x', y') - I(x + x', y + y'))^2 \]
  \item \texttt{CV_TM_SQDIFF_NORMED}
    \[R(x, y) = \cfrac{\sum_{(x', y') \in T} (T(x', y') - I(x + x', y + y'))^2}
                     {\sqrt{\sum_{(x', y') \in T} T(x', y')^2 \cdot
                            \sum_{(x', y') \in T} I(x + x', y + y')^2}} \]
  \item \texttt{CV_TM_CCORR}
    \[R(x, y) = \sum_{(x', y') \in T} (T(x', y') \cdot I(x + x', y + y'))^2 \]
  \item \texttt{CV_TM_CCORR_NORMED}
    \[R(x, y) = \cfrac{\sum_{(x', y') \in T} (T(x', y') \cdot I(x + x', y + y'))^2}
                      {\sqrt{\sum_{(x', y') \in T} T(x', y')^2 \cdot
                             \sum_{(x', y') \in T} I(x + x', y + y')^2}} \]
\end{itemize}

As the four formulas above show, we have, in fact, only two types of metrics,
the other two being only the normalized forms of the formers. The first two,
\texttt{CV_TM_SQDIFF_NORMED} and \texttt{CV_TM_SQDIFF} rely on the squared
difference between the value in the template image and the original image. If
we think of what a good match would mean in this case, namely when the two
pixels are as close as possible, than it means that the smaller the value is,
the better the match we have. So, in the case of the first two methods, all we
have to do is look for the smallest value in the result image in order to
find the best match.

\fig[scale=1]{src/img/template-result.jpg}{img:result-template}{Result matrix
\texttt{R} obtained with a metric of \texttt{CV_TM_CCORR_NORMED}.}

In \labelindexref{Figure}{img:result-template} we illustrate the result matrix
obtained after the sliding procedure shown in
\labelindexref{Figure}{img:sliding-template} using the
\texttt{CV_TM_CCORR_NORMED} metric. Contrary to the squared difference
metrics, the cross correlation methods yield the best result when the result
values is at its maximum, in our case the small white spot inside the red
circle. Once we have determined the location of the best match, all that
remains to be done is determine the contour of the template in the original
image. This can be easily achieved knowing the size of the template and the
fact that the white spot represents the upper left corner.

It is true that we presented here only the happy case when the template size
matches exactly the size of the object looked for in the query image which is
not very often true in a real environment where we know the size of the
template, but we know nothing about the bigger image. However, there is some
work done in this domain \cite{fast-ccorr} \cite{phase-angle} such that the
template matching approach can be used in real applications, like the one
presented in this thesis: recognizing logos.

Given its nature, the template matching technique is mainly used in situations
where the object characteristics are rather constant or where speed is more
important than very high accuracy. If we remember the numbers stated in
\labelindexref{Section}{sec:proj-motivation} from \labelindexref{Chapter}{chapter:intro},
it is not a surprise that this type of algorithm was the main choice in the implementation.

\todo{Decide whether we should talk about the Pyramid stuff or not}

\subsection{Learning Approach}
\todo{Present this type of technique. Implementations. Usages.
Advantages/Disadvantages. Performance blah blah}
