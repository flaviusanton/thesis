\chapter{Introduction}
\label{chapter:intro}

\todo{Talk about Social Media and analytics and importance of brand
recognition}

\section{Motivation}
\label{sec:proj-motivation}

Social Media has become an integral part of modern society, thus many
activities in the software industry emerge around it. In the last few years,
it has been observed that a social network like
Facebook or Twitter started to overcome their initial
purpose of easing access to information sharing or connecting people in the
online environment. One relatively new purpose of social networks is using
them as an advertisement platform. This \textit{use case} is not that hard to
imagine if we make a reference to the latest digits that social networks have
to bear these days.

According to an online statistics portal\cite{statista} Twitter has passed 300\footnote{\url{http://www.statista.com/statistics/272014/global-social-networks-ranked-by-number-of-users/}}
million monthly active users whereas Facebook is just 50 million users under
the 1.5 billion barrier, which is roughly one fifth of the total world population.
Needless to say, Instagram, Google+ or even Tumblr are over the 200
million monthly active users barrier. Moreover, there are 350 million new
photos uploaded every day on Facebook\footnote{\url{www.businessinsider.com/facebook-350-million-photos-each-day-2013-9}}
and 70 million on Instagram\footnote{\url{https://instagram.com/press/}}.
Twitter also faces more than 600 million tweets in a normal day of which
approximately 172\footnote{Numbers determined
experimentally while working on this project} million contain images.

With these rapidly increasing numbers, naturally comes a new area of
development and research called \textit{social media analytics}, which is "the
practice of gathering data from blogs and social media websites and analyzing
that data to make business decisions. The most common use of social media
analytics is to mine customer sentiment in order to support marketing and
customer service
activities"\footnote{\url{http://searchbusinessanalytics.techtarget.com/definition/social-media-analytics}}.
Understanding social media analytics gives companies a glimpse about how their
products and services are perceived by clients and potential clients. For
example, most of the people using a social network share content about their
own personal life, their thoughts, images and videos from their everyday life.
All this content can be used by companies to analyze how well they are
performing and try to improve their services (i.e. offer better advertisements
based on that user generated content).

That being said, in the last few years there was an increasing demand in the
industry for tools of analyzing unstructured data found on social networks.
These tools are specifically designed and built to consume as much data as they
can from the social platforms, process it and offer a variety of
representations (such as graphs, charts or tables) with aggregated and
meaningful results. In this way, a client company not only can find out how
much user reach it has, but also the general feeling about their services. For
example, the presence of a company logo in a tweet or Facebook post is a
strong indication that the user made a reference to a service or product
offered by the respective company, such that an analytics tool that has a
NLP (Natural Language Processing) component will feed that component with
additional information (i.e. it is very likely that the text is related to our
company) so it can be more effective.

\section{Objectives}

In this thesis we propose a system that focuses on logo mentions on social
networks streams. Although the tool is designed to work and run independently,
its major power would be paired with an existing social media analytics
platform, like the one that uberVU via
Hootsuite\footnote{\url{https://hootsuite.com/products/social-media-analytics/ubervu-via-hootsuite}}
offers. The project was designed and built from scratch in order to enrich
their existing tools by adding statistics about logo mentions and was done in
collaboration with engineers from Hootsuite.

One of the main objectives is to offer a very scalable system that accepts
multiple data sources (social media streams), identifies posts with images and
process those images in order to find logos, all in real-time. Before going
next, let us be clear with two apparently exhaustive terms used here:
\textit{very scalable} and \textit{real-time}. By a very scalable
system we understand a software program that, if offered
N\footnote{Where N is an arbitrarily chosen number}\% more
machines (processing nodes), will perform approximately N\% faster.
Of course, we say \textit{approximately} because we cannot ignore Amdahl's
Law\footnote{\url{http://home.wlu.edu/~whaleyt/classes/parallel/topics/amdahl.html}}.
Probably not as intuitive, by \textit{real-time} we understand, in this case, a
system that can fetch and process data from a streaming endpoint of a social
network without falling behind. To be even more clear, if Twitter offers
1\% of total tweets, that gives us approximately 6 million tweets to
process in a day. A simple math yields that this system has to process around
420 tweets per minute, or, more interestingly, 30 images per second only from
Twitter.

Another objective of this project is to build a modular system with loosely
coupled components, so that a new component can be inserted into the pipeline
at any future time without too much effort. With this in mind, this system is
designed as a chain of smaller self contained systems that can function
independently and only require an input stream of data and produce
an output stream of data. For example, one component can be the one that
fetches data from the social networks and outputs only relevant posts, in our
case the ones that contain images. The key objective here is that we can add
a new component in-between two existing ones without changing them.

Coupled with the above statements, the system not only has to identify logos
in downloaded images, but it can do any type of image processing. Although it
may seem so, the primary focus of this project was not building a tool that
only recognizes logos in social media streams, but a tool that can do any type
of processing with social media streams, providing it has a dedicated
component that does the work. Because we obviously needed a component of this type
in order to have a fully functional pipeline and because it is indeed useful
for social media analytics, a large part of this thesis concentrates on the
logo detection component, but the idea here is that the bigger purpose is the
actual pipelined infrastructure.

\section{Use Cases}
\label{sec:proj-use-cases}

From the user perspective, we designed the system far from being verbose,
the only visible output that it produces being a real time web dashboard that
offers various statistics about detected logos built on top of annotated data
stored in a database that gets polled once in a while. Nothing very
complicated, all the magic happening all the way from fetching web streams to
storing identified data in the database, but it is supposed to be like that,
the system aiming to be a smaller part in a large dashboard for a complete
analytics tool. 

Another use case is, for example, to couple this system (or even add a
separate component) with another one that does object or scene recognition, so
that the upper layers of analytics have a better insight on the context that
the logo was identified. It is clearly an advantage to know that Coca Cola's
logo appeared on a 330ml can that takes 50\% of the image or on some random
guy's T-shirt that points a bended iPhone 6\footnote{\url{http://goo.gl/2w3Lwz}}
to the camera and looks angry.

Along these lines, we can also imagine a company that recently launched a new
product and wants to know how much buzz has created around that product,
especially on social networks. It is true that there are many factors that
matter here, like user sentiment analysis, for example, but if we only see a
recent spike in the number of company logo occurrences on social media
streams, it can be an indicator that people are talking about it, that they
know about the product. Of course, it is not enough to have a decisive
conclusion, but combined with other tools we can definitely have richer
results.

From the technical point of view, we designed the system to have multiple,
usually replicated, components that run as background jobs and do not do any
interaction with a human user. Communication between components is done using
JSON objects, such that the components are completely decoupled. This
facilitates running the whole system in a cloud environment, like Amazon
EC2\footnote{\url{https://aws.amazon.com/ec2/}}. So, using the system consists
in starting some jobs and using the graphical interface that polls a database
that gets updated every time we have some new results (identified logos in our
case).

\section{Thesis Summary}

We have seen in the previous three sections what was the motivation of
building this project, what objectives we concentrated on and we also saw
a couple of situations where it would be useful to have a logo recognition
system.

In the next chapter, \textit{Related Work}, we talk about some
interesting algorithms and ideas that are currently emerging in the logo
recognition world. We explain that logos have very specific features that
sometimes may seem useful (i.e. most of them have 2-3 colors and those are
usually primary colors), but in fact are rather unhelpful.

In the third chapter we present the detailed implementation of the system,
with deep insight in every component as well as some general facts about a
messaging system used to pass around information between components. In
addition, we talk briefly about Scala, the programming language used in our
implementation and we offer a detailed description of the algorithm used
for logo recognition.

The fourth chapter focuses on testing and evaluation of current solution. We
will crunch some numbers regarding performance, machine costs, scalability or
even a precision and recall comparison.

The last two chapters present the conclusions and some ideas of future improvements to the present
implementation, not only from the algorithmic point of view, but also
regarding the underlying infrastructure.
