\chapter{Further Work}

This chapter provides an overview of the current lacks of the system and
possible future enhancements. Given the project nature, we have to take into
consideration both the logo recognition algorithm as a separate entity that
can be developed completely separately in an isolated environment and the
underlying infrastructure that supports everything.

\section{System Architecture Enhancements}

The first improvement that can be made would be to replicate the Fetcher
Module presented in \labelindexref{Section}{sub-sec:fetch-module}. This would
not introduce a great benefit in the present situation, but in the future
could prevent losing data in the event of a hardware or software failure that
would affect the Fetcher. It is important to understand that currently the
Fetcher is a single point of failure for this system and, if it fails, all
other components behind it will starve after a period of time. The obvious
solution to this problem is to have at least another replica that can come in and
take over the job of the failed Fetcher.

The second \textit{todo} on the list of improvements is, of course, supporting
more social networks, because, right now, Twitter is the only source of data.
Adding Facebook or Instagram would definitely increase the relevance of the
statistics we produce, but would also require a considerable amount of
additional processing nodes.

In terms of storage, a \textit{nice-to-have} feature is to move everything we
store on disk right now, especially downloaded images, on a cloud service that
is reliable and integrated in the cloud infrastructure that we are using. One
example of this technology is Amazon
S3\footnote{\url{https://aws.amazon.com/s3/}}.

Last, but not least, the Annotated Data Storage Module (see
\labelindexref{Section}{sub-sec:ads-module}) is also a single point of failure
in the system that needs to be replicated in the future. Fortunately, there is
very little traffic that reaches this point, so we can safely rely on the
Kafka log caching for now.

\section{Better Logo Recognition Algorithm}

A large amount of the time available was spent on building the pipelined
system that is basically the supporting infrastructure for image processing
algorithms that run inside the Detectors. Therefore, the accent has not been
on building the best possible logo recognition algorithm, but a
\textit{good enough} one that can prove the functionality of the underlying
system.

With this in mind, a decent improvement for the present implementation would
be a faster algorithm. It is hard to imagine a ground breaking improvement
while still using the template matching approach, but others could make a
difference. At the amount of machines that we currently need, even a 10\%
faster algorithm will result in about \$2000 saved per month, if we are
running at full Fetcher capacity.
